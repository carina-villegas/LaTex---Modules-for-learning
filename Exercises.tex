\documentclass{article}
\usepackage[utf8]{inputenc}
\usepackage{parskip}
\usepackage{hyperref}

\title{\textbf{Ejercicios}}
\author{Ing. Amb. Carina Villegas}

\begin{document}
\maketitle
\section{Módulo I y II}
\textbf{Ejercicio 1.} Crear un documento tipo artículo con el siguiente texto: Un documento .tex consta de dos partes, el preámbulo y el cuerpo.

\textbf{Ejercicio 2.} Introduce diversas secciones en el texto, correspondientes a las líneas aisladas del mismo. Prueba con diferentes clases de documentos: \textit{article, report, book} y presta atención a la numeración de las secciones para las distintas clases. Observa qué sucede si usas \textbackslash chapter con article.

\textbf{Ejercicio 3.} Al documento creado en el Ejercicio 2, señalar como tipo de documento \textit{article} y colocar: (a) un título y subtítulo cualquiera, (b) nombre del autor, (c) correo del autor, (d) Fecha, (e) Abstract y Resumen (en caso de tener problemas, resuelva mediante este \textbackslash renewcommand\textbackslash abstractname{Resumen}, finalmente (e) seccionar y subseccionar el documento. ¿Qué observaciones se puede destacar?. 

\textbf{Ejercicio 4.} Al documento creado en el Ejercicio 2, señalar como tipo de documento \textit{book} y colocar: (a) un título, (b) nombre del autor, (c) Fecha, y dividir el documento en capítulos y un apartado para referencias. 

\textbf{Ejercicio 5.} Crear una presentación (\textit{beamer}) compuesta por tres slides. El primer slide debe llevar sus nombres, la fecha y el título de su presentación (cualquiera), el segundo slide, una tabla de contenidos, y el tercer slide, debe contener el desarrollo del primer tema dispuesto en la tabla de contenidos.

\textbf{Ejercicio 6.} Esquematizar un documento tipo \textit{letter} con los siguientes comandos: \textbackslash signature, \textbackslash address, \textbackslash begin\{letter\}, \textbackslash opening\{Estimado/a,\},\textbackslash dots, \textbackslash closing\{Despedida\}, \textbackslash ps\{posdata\}, \textbackslash end\{cierre\}, \textbackslash end\{letter\}

\section{Módulo III}
\vspace{0.5cm}

The contribution of surface evapotranspiration (ET) to moist convection, cloudiness, and precipitation along the eastern flanks of the tropical Andes (EADS) was investigated using the Weather Research and Forecasting (WRF) Model with nested simulations of selected weather conditions down to 1.2-km grid spacing. To isolate the role of surface ET, numerical experiments were conducted using a quasi\-idealized approach whereby,at every time step,the surface sensible heat effects are exactly the same as in the reference simulations, whereas the surface latent heat fluxes are prevented from entering the atmosphere. Energy balance analysis indicates that surface ET influences moist convection primarily through its impact on conditional instability, because it acts as an important source of moist entropy in this region.The energy available for convection decreases by up to approximately 60\% when the ET contribution is withdrawn. In contrast, when convective motion is not thermally driven or under conditionally stable conditions, the role of latent heating from the land surface becomes secondary. At the scale of the Andes proper, removal of surface ET weakens upslope flows by increasing static stability of the lower troposphere, as the vertical gradient of water vapor mixing ratio tends to be less negative. Consequently, moisture convergence is reduced over the EADS. In the absence of surface ET, this process operates in concert with damped convective energy, suppressing cloudiness and decreasing daily precipitation by up to around 50\% in the simulations presented here.


\url{https://www.overleaf.com/project/60cffba1c569a4223d553f25}

\textbf{Ejercicio 1.} Considerando el texto dispuesto al inicio de esta apartado, colocar en negrita todas las cantidades porcentuales dadas en el texto, en itálica el texto dispuesto en paréntesis, cambiar el tamaño de letra a enorme de la primera palabra del párrafo y a muy pequeña todas las palabras "surface".

\textbf{Ejercicio 2.} Al texto dado previamente, dividir en dos secciones, cada con la denominación Parte 1 y Parte 2, respectivamente, alinear la primera sección a la derecha y la segunda sección a la izquierda. Observa qué sucede si en su lugar pones \textbackslash section*.

\textbf{Ejercicio 3.} Colocar la segunda sección en una nueva página, enfatizar la palabra \textit{Energy}, además, a la última oración de esta sección espaciar verticalmente 2 centímetros, disponer esta oración de color verde.

\textbf{Ejercicio 4.} Utilizar los siguientes comandos con el texto de la primera sección. ¿Puedes notar la diferencia entre ellas?
\begin{enumerate}
    \item \textbackslash begin \{quote\}...\textbackslash end\{quote\}
    \item \textbackslash begin \{quote\}...\textbackslash end\{quote\}
    \item \textbackslash begin \{verse\}...\textbackslash end\{verse\}
\end{enumerate}

\textbf{Ejercicio 5.} Introducir en un lugar de su elección una nota de pie y de margen de página, el siguiente texto: El comando para introducir una nota de pie página es footnote.

\end{document}
