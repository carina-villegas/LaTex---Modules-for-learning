\documentclass{beamer}
\usepackage[utf8]{inputenc}
\usepackage{graphicx}% incluir imágenes
\usepackage{ragged2e}% alineación de texto
\usepackage{hyperref}% incluir referencias y links
\usepackage{multicol}% incluir tablas
\usepackage{verbatim}
\usetheme{Copenhagen}
\title{Curso \LaTeX{} desde 0}
\author{Ing. Amb. Carina Villegas}
\date{Julio 2021}

\AtBeginSection
 { 
\begin{frame} 
  \frametitle{Tabla de contenidos}
  \tableofcontents[currentsection]
\end{frame}
}

\AtBeginSubsection{ 
\begin{frame}
  \frametitle{Tabla de contenidos}
  \tableofcontents[currentsection,currentsubsection]
\end{frame}
}

\begin{document}
\frame{\titlepage}

\section{MÓDULO II: Sintaxis básica}
\subsection{Caracteres especiales}
\begin{frame}{Caracteres especiales}
Cada carácter posee un significado, son los siguientes:
\begin{figure}
    \centering
    \includegraphics[width=1 \textwidth]{Graphics/Carateres especiales.PNG}
    \caption {Caracteres especiales para compilador \TeX{}}
    \label{fig:my_label}
\end{figure}
\end{frame}

\begin{frame}{Caracteres especiales}
   Para obtener los meros signos anteponemos \textbackslash
 \begin{figure}
     \centering
     \includegraphics[width=.35\textwidth]{Graphics/Signos.PNG}
     \caption{Signos}
     \label{fig:my_label}
 \end{figure}
\end{frame}

\subsection{Partes de un archivo .tex}
\begin{frame}{Partes de un archivo .tex}
\justifying
Todo documento poseerá dos partes, el \textbf{preámbulo}, donde se especificará el tipo de documento, tipo de letra, la inclusión de figuras, tablas o referencias, etc., y el \textbf{cuerpo} que corresponderá a nuestro contenido.\\
\end{frame}

\begin{frame}{Partes de un archivo .tex}
    \begin{block}{}
\hspace{2 cm}\textcolor{green}{Preámbulo}\\
\textbackslash documentclass\{tipo de documento\}\\
\textbackslash usepackage[utf8]\{inputenc\}\\ %Preámbulo principal. Codificación de caracteres más común en la red e inputec permite tener internamente el texto.
\textbackslash usepackage\{graphicx\} or \{subfigure\}\\% incluir imágenes there is a difference between graphics. graphicx is an extension of graphics. graphicx allows optional arguments according to the more transparent key=value scheme.
\textbackslash usepackage\{ragged2e\}\\% alineación de texto
\textbackslash usepackage\{hyperref\} o \{url\}\\% incluir referencias 
\textbackslash usepackage\{multirow\}\\% incluir tablas
\vspace{5 mm}
\hspace{2 cm}\textcolor{green}{Cuerpo}\\
\textcolor{red}{\textbackslash begin}\{document\}\\
    \hspace{1 cm}\textit{Cuerpo del documento}\\
\textcolor{red}{\textbackslash end}\{document\}\\
    \end{block}
\end{frame}

\begin{frame}{Más sobre el preámbulo}
\begin{enumerate}
    \item \textit{Idioma:} \textbackslash usepackage[spanish]\{babel\} %babel permite elegir el idioma
    \item \textit{Modo matemático:} amsmath, amssymb y amsfont.% todos, creados por la American Mathematical Society y amplían el catálogo de símbolos y entornos disponibles para escribir fórmulas matemáticas.
    \item \textit{Párrafos:} \textbackslash usepackage\{lipsum\} %párrafos aleatorios
    \item \textit{Matrices:} \textbackslash usepackage\{array\} % modo matemático y otras funciones matemáticas
    \item \textit{Modificar características de la página:} \textbackslash usepackage\{geometry\}
    \item \textit{Situación del texto o estructuras:} \textbackslash usepackage\{float\}
    \item \textit{Columnas:} \textbackslash usepackage\{multicol\}
    \item \textit{Enumeraciones:} \textbackslash usepackage\{enumerate\}
\end{enumerate}
\end{frame}

\subsection{Tipos de documento}
\begin{frame}{Tipos de documento}
\begin{columns}
\column{.5\textwidth}
\includegraphics[width=1.3\textwidth]{Graphics/Book.PNG}
\column{.45\textwidth}
\fbox{\textit{article:} artículo}
\vspace{2 mm}
\fbox {\textit{report:} memorias, proyectos,...}
\vspace{2 mm}
\fbox {\textit{book:} libros }
\vspace{2 mm}
\fbox {\textit{letter:} cartas }
\vspace{2 mm}
\fbox {\textit{slides:} Presentaciones tipo PP}
\vspace{2 mm}
\fbox {\textit{beamer:} Presentación}
\end{columns}
\end{frame}

\subsection{Ejercicios} 
\begin{frame}
    Carina Villegas Lituma\\
    
    \textit{INGENIERA AMBIENTAL}\\
    
    Correo: taniac.villegasl@gmail.com\\
    
\end{frame}

Referencias:
https://www.ucm.es/data/cont/docs/1346-2019-04-12-BaSix%20LaTeX%20ba%CC%81sico%20con%20ejercicios%20resueltos%20-%20Noir16.pdf
documento descargado como LaTex
\end{document}
